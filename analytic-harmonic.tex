\section{Analytic and Harmonic Functions}
\subsection{Boundary Values}
\begin{definition}[Poisson integral of some function or measure]
    Let $\tilde{f} : \D \to \C$ be a harmonic function. Then $f$ is said to be the \textit{Poisson integral} of the function $f : \T \to \C$ if
    \begin{equation*}
	\tilde{f} (re^{i\theta}) = \frac{1}{2\pi} \int_{T} f\left( e^{it} \right) P_{r} \left( e^{i\left( \theta-t \right)} \right) dt
    \end{equation*}
    Similarly, $f$ is said to be the \textit{Poisson integral} of a complex measure $\mu$ on $T$ if
\begin{equation*}
    \tilde{f} (re^{i\theta}) = \frac{1}{2\pi} \int_{T} P_{r} \left( e^{i\left( \theta-t \right)} \right) d\mu\left( e^{it} \right)
    \end{equation*}
    \label{def:Poisson-Integral-Of-Some-Function-Or-Measure}
\end{definition}
\subsection{Fatou's Theorem}

\begin{theorem}
    Let $\mu$ be a complex measure on the unit circle $\T$, and let $f: \D \to \C$ be the harmonic function defined by
    \begin{align*}
	f(re^{i\theta}) = \frac{1}{2\pi} \int_{\T} P_{r} \left( e^{i\left( \theta-t \right)} \right) d\mu \left( e^{it} \right)
    \end{align*}

    Let $e^{i\theta_{0}}$ be any point where $\mu$ is differentiable with respect to the normalised Lebesgue measure. Then
    \begin{equation*}
	\lim_{r\to 1} f\left( re^{i\theta_{0}} \right) = \left( \frac{d\mu}{d\theta} \right) \left( e^{i\theta _{0}} \right) = \mu ' \left( e^{i\theta _{0}} \right)
    \end{equation*}
    In fact, $f(re^{i\theta}) \to \mu ' \left( e^{i\theta_{0}} \right)$ as $re^{i\theta}$ approaches $e^{i\theta_{0}}$ along any path in the open disc within the region of the form $\abs{\theta - \theta_{0}} \le c \left( 1-r \right)$ for some $c> 0$. 
    \label{thm:Fatou-1906}
\end{theorem}


