\documentclass[12pt]{article}
\usepackage[margin=1in]{geometry}
\usepackage{amsfonts, amsmath}
\usepackage[T1]{fontenc}
\usepackage{mathrsfs, enumitem}
\usepackage{hyperref}
\usepackage[utf8]{inputenc}
\usepackage{amssymb}
\usepackage{amsfonts}
\usepackage{amsmath}
\usepackage{amsthm}
\usepackage{color}
\usepackage{hyperref}
\usepackage{csquotes}
%\usepackage{fourier}
\usepackage{tikz-cd}
\usepackage{lipsum}

\newtheorem{theorem}{Theorem}[subsection]
\newtheorem{lemma}[theorem]{Lemma}
\newtheorem{claim}[theorem]{Claim}
\newtheorem{proposition}[theorem]{Proposition}
\newtheorem{corollary}[theorem]{Corollary}
\newtheorem{fact}[theorem]{Fact}
\newtheorem{notation}[theorem]{Notation}
\newtheorem{observation}[theorem]{Observation}
\newtheorem{conjecture}[theorem]{Conjecture}
\newtheorem{exercise}[theorem]{Exercise}

\theoremstyle{definition}
\newtheorem{definition}[theorem]{Definition}
\newtheorem{example}[theorem]{Example}
\numberwithin{equation}{subsection}

\theoremstyle{remark}
\newtheorem{remark}[theorem]{Remark}
\theoremstyle{plain}
\newcommand{\ignore}[1]{}

% section symbol
\renewcommand{\thesection}{\S\arabic{section}}

% \renewcommand{\Pr}{{\bf Pr}}
% \newcommand{\Prx}{\mathop{\bf Pr\/}}
% \newcommand{\E}{{\bf E}}
% \newcommand{\Ex}{\mathop{\bf E\/}}
% \newcommand{\Var}{{\bf Var}}
% \newcommand{\Varx}{\mathop{\bf Var\/}}
% \newcommand{\Cov}{{\bf Cov}}
% \newcommand{\Covx}{\mathop{\bf Cov\/}}

% shortcuts for symbol names that are too long to type
\newcommand{\eps}{\epsilon}
\newcommand{\lam}{\lambda}
\renewcommand{\l}{\ell}
\newcommand{\la}{\langle}
\newcommand{\ra}{\rangle}
\newcommand{\wh}{\widehat}
\newcommand{\wt}{\widetilde}

% % "blackboard-fonted" letters for the reals, naturals etc.
\newcommand{\R}{\mathbb R}
\newcommand{\N}{\mathbb N}
\newcommand{\Z}{\mathbb Z}
\newcommand{\F}{\mathbb F}
\newcommand{\Q}{\mathbb Q}
\newcommand{\C}{\mathbb C}
\newcommand{\D}{\mathbb D}
\newcommand{\T}{\mathbb T}

% % operators that should be typeset in Roman font
% \newcommand{\poly}{\mathrm{poly}}
% \newcommand{\polylog}{\mathrm{polylog}}
% \newcommand{\sgn}{\mathrm{sgn}}
% \newcommand{\avg}{\mathop{\mathrm{avg}}}
% \newcommand{\val}{{\mathrm{val}}}

% % complexity classes
% \renewcommand{\P}{\mathrm{P}}
% \newcommand{\NP}{\mathrm{NP}}
% \newcommand{\BPP}{\mathrm{BPP}}
% \newcommand{\DTIME}{\mathrm{DTIME}}
% \newcommand{\ZPTIME}{\mathrm{ZPTIME}}
% \newcommand{\BPTIME}{\mathrm{BPTIME}}
% \newcommand{\NTIME}{\mathrm{NTIME}}

% values associated to optimization algorithm instances
\newcommand{\Opt}{{\mathsf{Opt}}}
\newcommand{\Alg}{{\mathsf{Alg}}}
\newcommand{\Lp}{{\mathsf{Lp}}}
\newcommand{\Sdp}{{\mathsf{Sdp}}}
\newcommand{\Exp}{{\mathsf{Exp}}}

% if you think the sum and product signs are too big in your math mode; x convention
% as in the probability operators
\newcommand{\littlesum}{{\textstyle \sum}}
\newcommand{\littlesumx}{\mathop{{\textstyle \sum}}}
\newcommand{\littleprod}{{\textstyle \prod}}
\newcommand{\littleprodx}{\mathop{{\textstyle \prod}}}

% horizontal line across the page
\newcommand{\horz}{
\vspace{-.4in}
\begin{center}
\begin{tabular}{p{\textwidth}}\\
\hline
\end{tabular}
\end{center}
}

% calligraphic letters
\newcommand{\calA}{{\cal A}}
\newcommand{\calB}{{\cal B}}
\newcommand{\calC}{{\cal C}}
\newcommand{\calD}{{\cal D}}
\newcommand{\calE}{{\cal E}}
\newcommand{\calF}{{\cal F}}
\newcommand{\calG}{{\cal G}}
\newcommand{\calH}{{\cal H}}
\newcommand{\calI}{{\cal I}}
\newcommand{\calJ}{{\cal J}}
\newcommand{\calK}{{\cal K}}
\newcommand{\calL}{{\cal L}}
\newcommand{\calM}{{\cal M}}
\newcommand{\calN}{{\cal N}}
\newcommand{\calO}{{\cal O}}
\newcommand{\calP}{{\cal P}}
\newcommand{\calQ}{{\cal Q}}
\newcommand{\calR}{{\cal R}}
\newcommand{\calS}{{\cal S}}
\newcommand{\calT}{{\cal T}}
\newcommand{\calU}{{\cal U}}
\newcommand{\calV}{{\cal V}}
\newcommand{\calW}{{\cal W}}
\newcommand{\calX}{{\cal X}}
\newcommand{\calY}{{\cal Y}}
\newcommand{\calZ}{{\cal Z}}

% bold letters (useful for random variables)
%----------------------------------------------
% \renewcommand{\a}{{\boldsymbol a}}
% \renewcommand{\b}{{\boldsymbol b}}
% \renewcommand{\c}{{\boldsymbol c}}
% \renewcommand{\d}{{\boldsymbol d}}
% \newcommand{\e}{{\boldsymbol e}}
% \newcommand{\f}{{\boldsymbol f}}
% \newcommand{\g}{{\boldsymbol g}}
% \newcommand{\h}{{\boldsymbol h}}
% \renewcommand{\i}{{\boldsymbol i}}
% \renewcommand{\j}{{\boldsymbol j}}
% \renewcommand{\k}{{\boldsymbol k}}
% \newcommand{\m}{{\boldsymbol m}}
% \newcommand{\n}{{\boldsymbol n}}
% \renewcommand{\o}{{\boldsymbol o}}
% \newcommand{\p}{{\boldsymbol p}}
% \newcommand{\q}{{\boldsymbol q}}
% \renewcommand{\r}{{\boldsymbol r}}
% \newcommand{\s}{{\boldsymbol s}}
% \renewcommand{\t}{{\boldsymbol t}}
% \renewcommand{\u}{{\boldsymbol u}}
% \renewcommand{\v}{{\boldsymbol v}}
% \newcommand{\w}{{\boldsymbol w}}
% \newcommand{\x}{{\boldsymbol x}}
% \newcommand{\y}{{\boldsymbol y}}
% \newcommand{\z}{{\boldsymbol z}}
% \newcommand{\A}{{\boldsymbol A}}
% \newcommand{\B}{{\boldsymbol B}}
% \newcommand{\C}{{\boldsymbol C}}
% \newcommand{\D}{{\boldsymbol D}}
% \newcommand{\E}{{\boldsymbol E}}
% \newcommand{\F}{{\boldsymbol F}}
% \newcommand{\G}{{\boldsymbol G}}
% \renewcommand{\H}{{\boldsymbol H}}
% \newcommand{\I}{{\boldsymbol I}}
% \newcommand{\J}{{\boldsymbol J}}
% \newcommand{\K}{{\boldsymbol K}}
% \renewcommand{\L}{{\boldsymbol L}}
% \newcommand{\M}{{\boldsymbol M}}
% \renewcommand{\O}{{\boldsymbol O}}
% \renewcommand{\P}{{\mathbb{P}}}
% \newcommand{\Q}{{\boldsymbol Q}}
% \newcommand{\R}{{\boldsymbol R}}
% \renewcommand{\S}{{\boldsymbol S}}
% \newcommand{\T}{{\boldsymbol T}}
% \newcommand{\U}{{\boldsymbol U}}
% \newcommand{\V}{{\boldsymbol V}}
% \newcommand{\W}{{\boldsymbol W}}
% \newcommand{\X}{{\boldsymbol X}}
% \newcommand{\Y}{{\boldsymbol Y}}
% \newcommand{\Z}{{\boldsymbol Z}}

% script letters
\newcommand{\scrA}{{\mathscr A}}
\newcommand{\scrB}{{\mathscr B}}
\newcommand{\scrC}{{\mathscr C}}
\newcommand{\scrD}{{\mathscr D}}
\newcommand{\scrE}{{\mathscr E}}
\newcommand{\scrF}{{\mathscr F}}
\newcommand{\scrG}{{\mathscr G}}
\newcommand{\scrH}{{\mathscr H}}
\newcommand{\scrI}{{\mathscr I}}
\newcommand{\scrJ}{{\mathscr J}}
\newcommand{\scrK}{{\mathscr K}}
\newcommand{\scrL}{{\mathscr L}}
\newcommand{\scrM}{{\mathscr M}}
\newcommand{\scrN}{{\mathscr N}}
\newcommand{\scrO}{{\mathscr O}}
\newcommand{\scrP}{{\mathscr P}}
\newcommand{\scrQ}{{\mathscr Q}}
\newcommand{\scrR}{{\mathscr R}}
\newcommand{\scrS}{{\mathscr S}}
\newcommand{\scrT}{{\mathscr T}}
\newcommand{\scrU}{{\mathscr U}}
\newcommand{\scrV}{{\mathscr V}}
\newcommand{\scrW}{{\mathscr W}}
\newcommand{\scrX}{{\mathscr X}}
\newcommand{\scrY}{{\mathscr Y}}
\newcommand{\scrZ}{{\mathscr Z}}

\newcommand{\im}{{\text{im }}}
\newcommand{\ip}[1]{\left\langle #1 \right\rangle}
\newcommand{\norm}[1]{\left\lVert #1 \right\rVert}
\newcommand{\abs}[1]{\left\lvert #1 \right\rvert}

\title{Banach Spaces of Analytic Functions}
\author{Ashish Kujur}
\date{Last Updated: \today}

\begin{document}

\maketitle \tableofcontents
\section{Analytic and Harmonic Functions}
\subsection{Boundary Values}
\subsubsection{Weak* convergence of measures}
\begin{theorem}
    Let $\left\{ \varphi _{i} \right\}_{i}$ be an approximate identity on $\T$ and let $\mu \in \calM \left( \T \right)$. Then for all $i$, $\varphi_{i} * \mu  \in L^{1} \left( \T \right)$ with
    \begin{equation*}
	\norm{\varphi _{i} * \mu}_{1} \le C_{\varphi} \lVert \mu \rVert
    \end{equation*}
    and
    \begin{equation*}
	\norm{\mu} \le \sup_{i} \norm{\varphi_{i} * \mu}_{1}\text{.}
    \end{equation*}
    Moreover, the measures $d\mu_{i} = \left( \varphi_{i} * \mu \right) \left( e^{it} \right) dt/2\pi$ converge to $d\mu \left( e^{it} \right)$ in the weak* topology, i.e.
    \begin{equation*}
	\lim_{i} \frac{1}{2\pi} \int_{-\pi}^{\pi} f\left( e^{it} \right) \left( \varphi_{i} * \mu \right) \left( e^{it} \right) dt = \int_{\T} \varphi\left( e^{it} \right) d\mu \left( e^{it} \right)
    \end{equation*}
    for all $f \in \calC \left( \T \right)$.
    \label{thm:weak-star-measures}
\end{theorem}

\subsubsection{Convergence in norm}
\begin{theorem}
    Let $\left\{ \varphi _{i} \right\}_{i}$ be an approximate identity on $\T$ and let $f \in L^{p} \left( \T \right)$ with $p \in [1, \infty)$. Then for all $i$, $\varphi_{i} * f  \in L^{p} \left( \T \right)$ with
    \begin{equation*}
	\norm{\varphi _{i} * f}_{p} \le C_{\varphi} \lVert f \rVert _{p}
    \end{equation*}
    and
    \begin{equation*}
	\lim_{i} \norm{\varphi_{i} * f -f}_{p} = 0 \text{.}
	    \end{equation*} 
    \label{thm:convergence-in-Lp}
\end{theorem}


\subsubsection{Weak* convergence of bounded functions}
\begin{theorem}
    Let $\left\{ \varphi _{i} \right\}_{i}$ be an approximate identity on $\T$ and let $f\in L^{\infty} \left( \T \right)$. Then for all $i$, $\varphi_{i} * \mu  \in \calC \left( \T \right)$ with
    \begin{equation*}
	\norm{\varphi _{i} * \mu}_{\infty} \le C_{\varphi} \lVert \mu \rVert _{\infty}
    \end{equation*}
    and
    \begin{equation*}
	\norm{f}_{+\infty} \le \sup_{i} \norm{\varphi_{i} * f}_{\infty}\text{.}
    \end{equation*}
    Moreover, $\varphi_{i} * f$ converge to $f$ in the weak* topology, i.e.
    \begin{equation*}
    \lim_{i} \int_{-\pi}^{\pi} g\left( e^{it} \right) \left( \varphi_{i} * f \right) \left( e^{it} \right) dt = \int_{\T} g\left( e^{it} \right) f \left( e^{it} \right) dt 
    \end{equation*}
    for all $g \in L^{1} \left( \T \right)$.
    \label{thm:weak-star-infinity}
\end{theorem}

\subsubsection{The entire picture!}

\begin{definition}[Poisson integral of some function or measure]
    Let $\tilde{f} : \D \to \C$ be a harmonic function. Then $\tilde{f}$ is said to be the \textit{Poisson integral} of the function $f : \T \to \C$ if
    \begin{equation*}
	\tilde{f} (re^{i\theta}) = \frac{1}{2\pi} \int_{T} f\left( e^{it} \right) P_{r} \left( e^{i\left( \theta-t \right)} \right) dt
    \end{equation*}
    In such a case, we will denote the function $\tilde{f}$ by $P[f]$.
    Similarly, $f$ is said to be the \textit{Poisson integral} of a complex measure $\mu$ on $T$ if
\begin{equation*}
    \tilde{f} (re^{i\theta}) = \frac{1}{2\pi} \int_{T} P_{r} \left( e^{i\left( \theta-t \right)} \right) d\mu\left( e^{it} \right)
    \end{equation*}In such a case, we will denote the function $\tilde{f}$ by $P[\mu]$.
    \label{def:Poisson-Integral-Of-Some-Function-Or-Measure}
\end{definition}

\begin{theorem}[Ultimate Convergence]
    Let $f : \D \to \C$ be a harmonic function. Define for each $r\in [0,1)$, the function $f_{r} : \T \to \C$ by
    \begin{equation*}
	f_{r} \left( e^{i\theta} \right) = f\left( re^{i\theta} \right)
    \end{equation*}
    The following statements holds:
    \begin{enumerate}
	\item If $1 < p \le \infty$ then $f=P[g]$ for some $g \in L^{p} [g]$ iff for each $r > 0$, $\norm{f_{r}}_{p} < +\infty$ .
	\item If p=1 then $f=P[g]$ for some $g \in L^{p} [g]$ iff $f_{r}$ converge in the $L^{1}$ norm.
	\item $f=P[\mu]$ for some $\mu \in \calM (\T)$ iff for each $r > 0$, $\norm{f_{r}}_{1} < +\infty$ 
    \end{enumerate}
    \label{thm:convergence-Poisson}
\end{theorem}
\subsection{Fatou's Theorem}

\begin{theorem}
    Let $\mu$ be a complex measure on the unit circle $\T$, and let $f: \D \to \C$ be the harmonic function defined by
    \begin{align*}
	f(re^{i\theta}) = \frac{1}{2\pi} \int_{\T} P_{r} \left( e^{i\left( \theta-t \right)} \right) d\mu \left( e^{it} \right)
    \end{align*}

    Let $e^{i\theta_{0}}$ be any point where $\mu$ is differentiable with respect to the normalised Lebesgue measure. Then
    \begin{equation*}
	\lim_{r\to 1} f\left( re^{i\theta_{0}} \right) = \left( \frac{d\mu}{d\theta} \right) \left( e^{i\theta _{0}} \right) = \mu ' \left( e^{i\theta _{0}} \right)
    \end{equation*}
    In fact, $f(re^{i\theta}) \to \mu ' \left( e^{i\theta_{0}} \right)$ as $re^{i\theta}$ approaches $e^{i\theta_{0}}$ along any path in the open disc within the region of the form $\abs{\theta - \theta_{0}} \le c \left( 1-r \right)$ for some $c> 0$. 
    \label{thm:Fatou-1906}
\end{theorem}

\begin{corollary}
    Let $\mu$ be a complex measure on $\T$. Then $P[\mu]$ has nontangential limits equal everywhere to the Radon Nikodym derivative of $\mu$ with respect to the normalised Lebesgue measure.
\end{corollary}

\begin{corollary}
    Let $f : \T \to \C$ be $L^{1}$. Then $P[f]$ has nontangential limits at almost everywhere and these limits equal to $f$ almost everywhere.
    \label{cor:L1-implies-Poisson-limits}
\end{corollary}

\begin{corollary}
    Let $f: \D \to \C$ be a harmonic function and $1\le p <\infty$. Suppose that for all $0\le r < 1$, we have that
    \begin{equation*}
	\norm{f_{r}}_{p} < +\infty
    \end{equation*}
    Then for almost every $\theta$ the radial limits 
    \begin{equation*}
	\tilde {f} (e^{i\theta} ) = \lim_{r\to 1} f\left( re^{i\theta} \right)
    \end{equation*}
    exist and define a function $\tilde f$ in $L^{p} \left( \T \right)$. The following also holds:
    \begin{enumerate}
	\item If $p>1$ then $f=P[\tilde{f}]$.
	\item If $p=1$ then $f=P[\mu]$ for some complex measure $\mu$ whose absolutely continuous part is $fd\theta$.
	\item IF $f$ is bounded then the boundary values exist almost everywhere and define a bounded measurable function $\tilde{f}$ on $\T$ such that $f=P[\tilde{f}]$.
    \end{enumerate}
    \label{cor:imp-Fatou}
\end{corollary}
\begin{proof}
    Suppose that for each $r\in [0,1)$, we have $\norm{f_{r}}_{p} < +\infty$. We need to prove that for almost every $\theta$, $\lim_{r\to 1} f\left( re^{i\theta} \right)$ exists. Then by Theorem \ref{thm:convergence-Poisson}, we have that $f=P[g]$ for some $g\in L^{p} \left( \T \right)$. Since $L^{p} \left( \T \right) \subset L^{1} \left( \T \right)$, we can use the previous corollary. By the previous corollary, we have that $P[g]$ has nontangential limits almost everywhere, we have that
    \begin{equation}
	\tilde{f} \left( e^{i\theta} \right) = \lim_{r\to 1} f(re^{i\theta}) = \lim_{r\to 1} P[g] \left( re^{i\theta} \right)
	\label{eqn:radial-limits}
    \end{equation}
    exists almost everywhere.

    Now we proceed to prove part $(1)$. Also by Theorem \ref{thm:convergence-Poisson}, we have that $f=P[g]$ for some $g\in L^{p} \left( \T \right)$. Hence, we have that by Equation \ref{eqn:radial-limits} that $\tilde{f} (e^{i\theta}) =  \lim_{r\to 1} P[g] \left( re^{i\theta} \right)$ holds at almost every $\theta$.

    Also, by the previous corollary, $\lim_{r\to 1} P[g] \left( re^{i\theta} \right) = g(e^{i\theta})$ for almost every $\theta$. Hence, we have that $\tilde{f} = g$.

\end{proof}

\begin{corollary}
    Let $f: \D \to \R_{\ge 0}$ be a harmonic function. Then $f$ has nontangential limits at almost every point of $\T$. \textcolor{red}{(Why demand nonnegative?)}
    \label{cor:nontangential-limits}
\end{corollary}

Let $h \left( \D \right)$ denote the set of all harmonic functions on $\D$. Let $p\in [1,\infty]$. Define
\begin{equation*}
    h^{p} \left( \D \right) = \left\{ f\in h \left( \D \right) \, \mid \, \left\{ f_{r} \right\}_{0\le r < 1} \text{ is uniformly bounded in } L^{p} \text{ norm }\right\}
\end{equation*}
We define a norm on $h^{p} \left( \D \right)$ by
\begin{equation*}
    \norm{f} = \sup_{0\le r < 1} \norm{f_{r}}_{p} =\begin{cases} \sup_{0\le r < 1} \left( \frac{1}{2\pi} \int_{-\pi}^{\pi} \abs{f\left( re^{i\theta}\right)}^{p}d\theta   \right)^{\frac{1}{p}} & \text{if } p \in [1, \infty) \\
	\sup_{0\le r < 1} \norm{f(re^{i\theta})}_{\infty}
    \end{cases}
\end{equation*}

It is easy to see why $\norm{f} < +\infty$ for any $f\in h^{p} \left( D \right)$. So we now proceed to show that $h^{p} \left( D \right)$ is a Banach space with this norm. First we show that it is indeed a normed linear space.

Clearly, $h \left( \D \right)$ is a vector space. To show that $h^{p} \left( \D \right)$ is a vector space, it suffices to check that $h^{p} \left( \D \right)$ is a subspace.

Let $f,g \in h^{p} \left( \D \right)$ and let $\alpha \in \C$. Then for any $r\in [0,1)$, we have that 
\begin{align*}
    \norm{(f+\alpha g)_{r}}_{p} &= \norm{f_{r} + \alpha g_{r}} \\
    &= \norm{f_{r}}_{p} + \alpha \norm{g_{r}}_{p}
\end{align*}
Take note of the use of Holder's inequality. After this is done, since $\left\{ f_{r} \right\}_{r\in [0,1)}$ and $\left\{ g_{r} \right\}_{r\in [0,1)}$ is uniformly bounded, we have that $\left\{ f+ \alpha g \right\}_{r\in [0,1)}$ is uniformly bounded in $L^{p}$ norm.

Now, we need to show that it is a normed linear space but this follows almost immediately.

To show that it is a Banach space, we show that

\begin{theorem}
    Let $p\in [1, \infty]$. If $u\in L^{p} \left( \T \right)$ then $f=P * u \in h^{p} \left( \D \right)$ and $\norm{f}_{p} =\norm{u}_{p}$. If $\mu \in \calM \left( \T \right)$ then $f= P * \mu \in h^{1} \left( \D \right)$ and $\norm{f}_{1} = \norm{\mu}$.
    \label{thm:lp-and-hp}
\end{theorem}
\begin{proof}
    We consider the case $p \in [1, \infty )$. The other cases can be dealt similarly. Consider the map 
    \begin{equation*}
	u \stackrel{T}{\mapsto} U
    \end{equation*}
    where $U\left( re^{i\theta} \right) = \frac{1}{2\pi} \int_{-\pi}^{\pi} P_{r}\left( e^{i(\theta-t)} \right) u \left( e^{it} \right) dt$. By Theorem \ref{thm:convergence-in-Lp}, we have that $\norm{U} = \norm{u}_{p} < +\infty$. Hence $U \in h^{p} \left( \D \right)$.

    Linearity is obvious. We need to check injectivity and surjectivity.

    To check injectivity, let $u \in L^{p} \left( \T \right)$ and suppose that $T(u)=P[u]=0$. Now $\lim_{r\to 1} P[u] \left( re^{i\theta} \right) = u$ for almost $\theta$ by Corollary \ref{cor:L1-implies-Poisson-limits} and hence $u=0$ almost everywhere.

    Surjectivity is clear from Theorem \ref{thm:convergence-Poisson}.
    \end{proof}


\subsection{\texorpdfstring{$H^p$}{\text{Hp}} spaces}

\section{The space \texorpdfstring{$H^1$}{\text{H1}}}
\horz

\subsection{Brief Recap!}
\begin{theorem}
    Let $u : \overline{\D} \to \C$ be a harmonic function. Then we have that
    \begin{equation*}
	u\left( re^{i\theta} \right) = \frac{1}{2\pi} \int_{-\pi}^{\pi} u\left( e^{it} \right) P_{r} \left( e^{i\left( \theta-t \right)} \right)
    \end{equation*}
    \label{thm:Poisson-Integral-Formula}
\end{theorem}

\subsection{The Helson-Lowdenslager Approach}
Let $\calC \left( \overline {\D} \right)$ be the set of all continuous functions on $\overline{\D}$ and let $H\left( \D \right)$ be the set of all holomorphic functions on the open disc $\D$. We define $\calA =\calC \left( \overline{ \D } \right) \cap H \left( \D \right)$.

We show that $\calA$ is an uniformly closed algebra of $\calC \left( \overline{ \D } \right)$. Let $\left\{ f_{n} \right\}$ be a sequence in $\calA$ converging uniformly to $f\in \calC \left( \overline \D \right)$.

We recall Morera's Theorem for analytic functions at this point:
\begin{theorem}[Morera]
    A continuous, complex valued function $f : D \to \C$ that satisfies $\oint _{\gamma} f \left( z \right) dz = 0$ for any closed piecewise $C^{1}$ path $\gamma$ in $D$ must be holomorphic on $D$.
    \label{thm:morera-analytic}
\end{theorem}

We use this theorem to prove what we want to prove. Now, let $C$ be any closed curve in $\D$. Then for any $n\in \mathbb N$,
\begin{align*}
\oint_{C}f_{n}  \left( z \right) dz = 0
\end{align*}
So, 
\begin{align*}
    \oint_{C} f(z) dz = \oint_{C} \lim_{n\to \infty} f_{n} \left( z \right) dz = \lim_{n\to \infty} \oint_{C} f_{n} \left( z \right) dz =0
\end{align*}
Since $C$ was arbitrary, $f$ must be holomorphic. This shows that $\calA$ is uniformly closed. The fact that it is an algebra is easy to check $\checkmark$.

Now, note that since $\D$ is a compact metric space, we have that $\calC \left( \D \right)$ is a complete metric space with supremum metric. Since the supremum metric can also be induced by a norm, namely the supremum norm, we have that $\calC\left( \D \right)$ is a Banach space with the supremum  norm.

Thus, this is what we have proved so far:

\begin{theorem}
    The disc algebra $\calA =\calC \left( \overline{ \D } \right) \cap H \left( \D \right)$ is a Banach space under the $\sup$ norm
    \begin{align*}
	\norm{f}_{\infty} = \sup_{\abs{z}\le 1} \abs{f\left( z \right)}
    \end{align*}
    \label{thm:disc-algebra-is-B-space}
\end{theorem}

We make a couple of observations at this point:
\begin{enumerate}
    \item Each $f\in \mathcal A$ is the Poisson integral of its boundary values:
\begin{align*}
    f\left( re^{i\theta} \right) = \frac{1}{2\pi} \int_{-\pi}^{\pi} f\left( e^{it} \right) P_{r} \left( e^{i \left( \theta -t \right)} \right) dt
\end{align*}
\item It follows from the Maximum Modulus Theorem that 
    \begin{align*}
	\norm{f}_{\infty} = \sup \abs{f\left( e^{it} \right)}
    \end{align*}
\end{enumerate}

\begin{theorem}[Correspondence of $\calA$ with a closed subspace of $\calC \left( \T \right)$]
    Consider the subspace
    \begin{align*}
	\tilde{\calA} = \left\{ f\in \calC \left( \T \right) \, : \, \int_{-\pi}^{\pi} f\left( e^{it} \right) e^{in\theta}  \text{ for } n=1,2,\ldots \right\}
    \end{align*}
    of $\calC \left( \T \right)$. Then there is an isomorphism of $\calA$ with $\tilde{\calA}$.
    \label{thm:correspondence-of-disc-algebra}
\end{theorem}
\begin{proof}
    First, we show that $\tilde{\calA}$ is a closed subspace of $\calC \left( \T \right)$. Let $\left\{ f_{n} \right\}$ be a sequence of functions in $\tilde{\calA}$ converging to $f\in \calC \left( \T \right)$. Consider the following:
    \begin{align*}
	\abs{\int_{-\pi}^{\pi} f\left( e^{it} \right) e^{ikt} dt} &= \abs{\int_{-\pi}^{\pi} f\left( e^{it} \right) e^{ikt} dt - \int_{-\pi}^{\pi} f_n\left( e^{it} \right) e^{ikt} dt}  \\
	&= \int_{-\pi}^{\pi} \abs{f \left( e^{it} \right) -f_{n} \left( e^{it} \right)} dt \\
	& \le 2\pi \norm{f_{n} - f}_{\infty} \to 0 \text{ as } n\to \infty
    \end{align*}
    This shows that $\tilde{\calA}$ is closed under $\calC \left( \T \right)$ with supremum norm.

    Now consider the linear map $T: \calA \to \tilde{\calA}$ given by
    \begin{align*}
	f \stackrel{T}{\longmapsto} f\mid_{\T}
    \end{align*}
    For the sake of convenience, we will write $f\mid_{\T}$ as $f_{\T}$.
    We first need to show this map is well defined! That is, we need to show that
    \begin{equation*}
	\int_{-\pi}^{\pi} f_{\T} \left( e^{it} \right) e^{ikt} dt = 0	
    \end{equation*}
    for all $k\in \N$ but this immediately follows from Cauchy's theorem.

    Note that injectivity is clear from Theorem \ref{thm:Poisson-Integral-Formula}. To show surjectivity, let $f\in \tilde{A}$. We need to show that there is a function $u \in \calA$ such that $u_{\T} = f$. Consider the function
    \begin{equation*}
	u\left( re^{i\theta} \right) =
	\begin{cases}
	    (P*f) (re^{i\theta}) & \text{ if } 0\le r <1 \\ 
	    f\left( e^{i\theta} \right) & \text{ if } r=1
	\end{cases}
    \end{equation*}
    This is the Dirichlet problem on the unit disc! So, $u$ is continuous on $\overline {\D}$. It remains to show that $u$ is analytic on $\D$. But note that for $r\in [0,1)$,
    \begin{align*}
    u\left( re^{i\theta} \right) &= \sum_{n=-\infty}^{\infty} r^{\abs{n}} \hat{f}\left( n \right) e^{int} \\
&= \sum_{n=0}^{\infty} r^{\abs{n}} \hat{f}\left( n \right) e^{int} 
    \end{align*}
    This completes the proof of the theorem!
\end{proof}

In view of the previous theorem, we will simply write $\tilde{\calA}$ as $\calA$.

\begin{definition}
    A trignometric polynomial $p$ on the circle $\T$ is of the form
    \begin{equation*}
	p\left( e^{it} \right) = \sum_{k=0}^{n} a_{k} e^{ik\theta}
    \end{equation*}
    \label{def:trignometric-polynomial}
\end{definition}

\begin{proposition}
    The set of the trignometric polynomials is a dense subset of $\calA$.
    \label{prop:disc-algebra-separable}
\end{proposition}
\begin{proof}
    It is clear that any trignometric polynomial on the circle is a member of the disc algebra. Now, if $f: \T \to \C$ is in $\calA$, then its negative Fourier coefficients are zero! Since, the Cesaro sum of $f$ 
    \begin{equation*}
	s_{n} \left( x \right) = \sum_{k=-n}^{k=n} \hat{f} (n) e^{ikx} = \sum_{k=0}^{n} \hat{f} (n) e^{ikx}
    \end{equation*}
    converge to $f$ uniformly and is a sequence of trignometric polynomial, we are done!
\end{proof}

The following result is used in the proof of the next theorem, so, we prove it here:

\begin{theorem}
    The real parts of functions in $\calA$ are uniformly dense in $\calC \left( \T , \R \right)$. In other words, if $\mu$ is finite signed Borel measure on $\T$ such that $\int f \, d\mu =0$ for every $f\in \calA$ then $\mu$ is the zero measure.
    \label{thm:kinda-Fejer}
\end{theorem}
\begin{proof}
    We first show that any trignometric polynomial of the form
    \begin{equation}
	p\left( e^{it} \right) = \sum_{k=-n}^{n} c_{k} e^{ikt}
	\label{eqn:trig-poly-1}
    \end{equation}
    where $c_{-k} = \overline{c_{k}}$ for each $k\in \left\{ 1,\ldots , n \right\}$ is a real part of a function $f\in \calA$. Note that $p\left( e^{it} \right)$ in Equation \ref{eqn:trig-poly-1} is the real part of the function:
    \begin{equation*}
	f\left( e^{it} \right) = c_{0} + 2c_{1} e^{it} + 2c_{2} e^{2it} + \ldots + 2 c_{n} e^{int}
    \end{equation*}

    
    Now, we claim that every function $f \in C\left( \T , \R \right)$ is a uniform limit of a trignometric polynomial of the form \ref{eqn:trig-poly-1}. We will be done if we show that the negative Fourier coefficients of real valued function is the conjugate of the its positive counterpart, that is, for each $n\in \Z_{\ge 0}$, we have that $\hat{f}\left( -n \right) = \overline{\hat{f} \left( n \right)}$. To show this, take any $n\in \Z _{\ge 0}$ and then observe that
    \begin{align*}
	\hat{f} \left( -n \right) &= \int_{\T} f\left( e^{it} \right) e^{int} \frac{dt}{2\pi} \\
	&= \overline{\int_{\T} f\left( e^{it} \right) e^{-int} \frac{dt}{2\pi}} \\
&= \overline{\hat{f} \left( n \right)}
    \end{align*}

This shows that the Cesaro means of a real valued function is a trignometric polynomial of the form \ref{eqn:trig-poly-1} and since the Cesaro means converges to $f$ uniformly. Thus, the closure of the real parts of $\calA$ is indeed $\calC \left( \T , \R \right)$.

Now, let $\mu$ be a finite signed measure on $\T$ such that $\int f \, d\mu =0$ for every $f \in \calA$. We show that $\mu$ is the zero measure. Notice that if $f\in \calA$ then
\begin{equation*}
    0=\int_{\T} f \, d\mu = \int_{\T} \Re (f) \, d\mu + i\int_{\T} \Im (f) \, d\mu
\end{equation*}
Hence, it follows that 
\begin{equation*}
    \int_{\T} \Re (f) \, d\mu = 0
\end{equation*}
for every $f\in \calA$.
Now, if $g\in \calC \left( \T, \R \right)$ then by the first part of this theorem, there is a sequence $\left\{ f_{n} \right\} \in \calA$ such that $\Re \left( f_{n} \right) $ converges to $g$ uniformly. By the Dominated Convergence Theorem (which holds, thanks to Jordan Decomposition Theorem), we have that $\int_{\T} g \, d\mu = 0$. 

Now to prove that every $\mu = 0$, it suffices to show that $\hat{\mu} \left( n \right) = 0$ for every $n\in \Z$\footnote{See Page 41, Corollary 2.3 of Mashreghi.}. Now, notice that for any $n\in \Z$, we have
\begin{align*}
    \hat{\mu} (n) &= \int_{\T} e^{-int} d\mu \left( e^{it} \right) \\
    &=  \int_{\T} \left(\cos \left( nt \right) - i \sin \left( nt \right) \right) d\mu \left( e^{it} \right) \\
    &= \int_{\T} \cos \left( nt \right) d\mu \left( e^{it} \right) - i\int_{\T} \sin \left( nt \right) d\mu \left( e^{it} \right) \\
    &= 0
\end{align*}
This completes the proof!
\end{proof}

\begin{corollary}
    Let $\mu$ be a finite signed  measure on $\T$ such that $\int f d\mu = 0$ for every $f\in \calA$ which vanishes at the origin then $\mu$ is a constant multiple of Lebesgue measure.
    \label{cor:constant-multiple-of-Lebesgue}
\end{corollary}
\begin{proof}
    We first prove the following claim: If $f \in \calA$ then $\int_{\T} f d\mu = \frac{f\left( 0 \right)}{2\pi}$. Since the negative Fourier cofficients are zero and $f$ is continuous, we have that the Cesaro means converge uniformly to $f$, that is,
    \begin{align*}
	\sum_{k=0}^{\infty} \hat{f} \left( n \right) e^{int} \rightarrow f \text{ uniformly}
    \end{align*}
    Thus,
\begin{align*}
    \int_{T} f\left( e^{it} \right) \frac{dt}{2\pi} &= \frac{1}{2\pi} \int_{\T}  \left(  \sum_{k=0}^{\infty}\hat{f} \left( n \right) e^{int} \right) dt\\
    &= \frac{1}{2\pi} \sum_{k=0}^{\infty} \int_{\T} \left( \hat{f} \left( n \right) e^{int} dt \right) \\
    &= \frac{f(0)}{2\pi}
    \end{align*}

    Now, we proceed to the proof. We define a measure $d\nu = d\mu - \frac{1}{2\pi} \mu \left( \T \right) dt$. Now, we have that
    \begin{align*}
	\int_{\T} f\left( e^{it} \right) d\nu \left( e^{it} \right) &= \int_{\T}[f-f\left( 0 \right)] \left( e^{it} \right) d\mu \left( e^{it} \right) + f\left( 0 \right) \int_{\T} d\nu \left( e^{it} \right) \\
	&= 0
    \end{align*}
    Hence, we have that $d\mu = \frac{1}{2\pi} \mu \left( \T \right) dt$.
\end{proof}

\horz 
%%%%%%%%%%%%%%%%%%%%%%%%%%%%%%%%%%%%%%%%%%%%%%%%%%%%%%%%%%%%%%%%%%%%%%%%%%%%%%%%%%%%%

\begin{theorem}[F and M. Riesz]
    
    \label{thm:f-m-riesz}
\end{theorem}

\subsection{Szegö's Theorem}

\end{document}
