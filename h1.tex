\chapter{The space \texorpdfstring{$H^1$}{\text{H1}}}

\section{Brief Recap!}
\begin{theorem}
    Let $u : \overline{\D} \to \C$ be a harmonic function. Then we have that
    \begin{equation*}
	u\left( re^{i\theta} \right) = \frac{1}{2\pi} \int_{-\pi}^{\pi} u\left( e^{it} \right) P_{r} \left( e^{i\left( \theta-t \right)} \right)
    \end{equation*}
    \label{thm:Poisson-Integral-Formula}
\end{theorem}

\section{The Helson-Lowdenslager Approach}
Let $\calC \left( \overline {\D} \right)$ be the set of all continuous functions on $\overline{\D}$ and let $H\left( \D \right)$ be the set of all holomorphic functions on the open disc $\D$. We define $\calA =\calC \left( \overline{ \D } \right) \cap H \left( \D \right)$.

We show that $\calA$ is an uniformly closed algebra of $\calC \left( \overline{ \D } \right)$. Let $\left\{ f_{n} \right\}$ be a sequence in $\calA$ converging uniformly to $f\in \calC \left( \overline \D \right)$.

We recall Morera's Theorem for analytic functions at this point:
\begin{theorem}[Morera]
    A continuous, complex valued function $f : D \to \C$ that satisfies $\oint _{\gamma} f \left( z \right) dz = 0$ for any closed piecewise $C^{1}$ path $\gamma$ in $D$ must be holomorphic on $D$.
    \label{thm:morera-analytic}
\end{theorem}

We use this theorem to prove what we want to prove. Now, let $C$ be any closed curve in $\D$. Then for any $n\in \mathbb N$,
\begin{align*}
\oint_{C}f_{n}  \left( z \right) dz = 0
\end{align*}
So, 
\begin{align*}
    \oint_{C} f(z) dz = \oint_{C} \lim_{n\to \infty} f_{n} \left( z \right) dz = \lim_{n\to \infty} \oint_{C} f_{n} \left( z \right) dz =0
\end{align*}
Since $C$ was arbitrary, $f$ must be holomorphic. This shows that $\calA$ is uniformly closed. The fact that it is an algebra is easy to check $\checkmark$.

Now, note that since $\D$ is a compact metric space, we have that $\calC \left( \D \right)$ is a complete metric space with supremum metric. Since the supremum metric can also be induced by a norm, namely the supremum norm, we have that $\calC\left( \D \right)$ is a Banach space with the supremum  norm.

Thus, this is what we have proved so far:

\begin{theorem}
    The disc algebra $\calA =\calC \left( \overline{ \D } \right) \cap H \left( \D \right)$ is a Banach space under the $\sup$ norm
    \begin{align*}
	\norm{f}_{\infty} = \sup_{\abs{z}\le 1} \abs{f\left( z \right)}
    \end{align*}
    \label{thm:disc-algebra-is-B-space}
\end{theorem}

We make a couple of observations at this point:
\begin{enumerate}
    \item Each $f\in \mathcal A$ is the Poisson integral of its boundary values:
\begin{align*}
    f\left( re^{i\theta} \right) = \frac{1}{2\pi} \int_{-\pi}^{\pi} f\left( e^{it} \right) P_{r} \left( e^{i \left( \theta -t \right)} \right) dt
\end{align*}
\item It follows from the Maximum Modulus Theorem that 
    \begin{align*}
	\norm{f}_{\infty} = \sup \abs{f\left( e^{it} \right)}
    \end{align*}
\end{enumerate}

\begin{theorem}[Correspondence of $\calA$ with a closed subspace of $\calC \left( \T \right)$]
    Consider the subspace
    \begin{align*}
	\tilde{\calA} = \left\{ f\in \calC \left( \T \right) \, : \, \int_{-\pi}^{\pi} f\left( e^{it} \right) e^{int}=0  \text{ for } n=1,2,\ldots \right\}
    \end{align*}
    of $\calC \left( \T \right)$. Then there is an isometric isomorphism of $\calA$ with $\tilde{\calA}$.
    \label{thm:correspondence-of-disc-algebra}
\end{theorem}
\begin{proof}
    First, we show that $\tilde{\calA}$ is a closed subspace of $\calC \left( \T \right)$. Let $\left\{ f_{n} \right\}$ be a sequence of functions in $\tilde{\calA}$ converging to $f\in \calC \left( \T \right)$. Consider the following:
    \begin{align*}
	\abs{\int_{-\pi}^{\pi} f\left( e^{it} \right) e^{ikt} dt} &= \abs{\int_{-\pi}^{\pi} f\left( e^{it} \right) e^{ikt} dt - \int_{-\pi}^{\pi} f_n\left( e^{it} \right) e^{ikt} dt}  \\
	&= \int_{-\pi}^{\pi} \abs{f \left( e^{it} \right) -f_{n} \left( e^{it} \right)} dt \\
	& \le 2\pi \norm{f_{n} - f}_{\infty} \to 0 \text{ as } n\to \infty
    \end{align*}
    This shows that $\tilde{\calA}$ is closed under $\calC \left( \T \right)$ with supremum norm.

    Now consider the linear map $T: \calA \to \tilde{\calA}$ given by
    \begin{align*}
	f \stackrel{T}{\longmapsto} f\mid_{\T}
    \end{align*}
    For the sake of convenience, we will write $f\mid_{\T}$ as $f_{\T}$.
    We first need to show this map is well defined! That is, we need to show that
    \begin{equation*}
	\int_{-\pi}^{\pi} f_{\T} \left( e^{it} \right) e^{ikt} dt = 0	
    \end{equation*}
    for all $k\in \N$ but this immediately follows from Cauchy's theorem.

    Note that injectivity is clear from Theorem \ref{thm:Poisson-Integral-Formula}. To show surjectivity, let $f\in \tilde{A}$. We need to show that there is a function $u \in \calA$ such that $u_{\T} = f$. Consider the function
    \begin{equation*}
	u\left( re^{i\theta} \right) =
	\begin{cases}
	    (P*f) (re^{i\theta}) & \text{ if } 0\le r <1 \\ 
	    f\left( e^{i\theta} \right) & \text{ if } r=1
	\end{cases}
    \end{equation*}
    This is the Dirichlet problem on the unit disc! So, $u$ is continuous on $\overline {\D}$. It remains to show that $u$ is analytic on $\D$. But note that for $r\in [0,1)$,
    \begin{align*}
    u\left( re^{i\theta} \right) &= \sum_{n=-\infty}^{\infty} r^{\abs{n}} \hat{f}\left( n \right) e^{int} \\
&= \sum_{n=0}^{\infty} r^{\abs{n}} \hat{f}\left( n \right) e^{int} 
    \end{align*}
    This completes the proof of the theorem!
\end{proof}

In view of the previous theorem, we will simply write $\tilde{\calA}$ as $\calA$.

\begin{definition}
    An analytic trignometric polynomial $p$ on the circle $\T$ is of the form
    \begin{equation*}
	p\left( e^{it} \right) = \sum_{k=0}^{n} a_{k} e^{ik\theta}
    \end{equation*}
    \label{def:trignometric-polynomial}
\end{definition}

\begin{proposition}
    The set of the trignometric polynomials is a dense subset of $\calA$.
    \label{prop:disc-algebra-separable}
\end{proposition}
\begin{proof}
    It is clear that any trignometric polynomial on the circle is a member of the disc algebra. Now, if $f: \T \to \C$ is in $\calA$, then its negative Fourier coefficients are zero! Since, the Cesaro sum of $f$ 
    \begin{equation*}
	s_{n} \left( x \right) = \sum_{k=-n}^{k=n} \hat{f} (n) e^{ikx} = \sum_{k=0}^{n} \hat{f} (n) e^{ikx}
    \end{equation*}
    converge to $f$ uniformly and is a sequence of trignometric polynomial, we are done!
\end{proof}

The following result is used in the proof of the next theorem, so, we prove it here:

\begin{theorem}
    The real parts of functions in $\calA$ are uniformly dense in $\calC \left( \T , \R \right)$. In other words, if $\mu$ is finite signed Borel measure on $\T$ such that $\int f \, d\mu =0$ for every $f\in \calA$ then $\mu$ is the zero measure.
    \label{thm:kinda-Fejer}
\end{theorem}
\begin{proof}
    We first show that any trignometric polynomial of the form
    \begin{equation}
	p\left( e^{it} \right) = \sum_{k=-n}^{n} c_{k} e^{ikt}
	\label{eqn:trig-poly-1}
    \end{equation}
    where $c_{-k} = \overline{c_{k}}$ for each $k\in \left\{ 1,\ldots , n \right\}$ is a real part of a function $f\in \calA$. Note that $p\left( e^{it} \right)$ in Equation \ref{eqn:trig-poly-1} is the real part of the function:
    \begin{equation*}
	f\left( e^{it} \right) = c_{0} + 2c_{1} e^{it} + 2c_{2} e^{2it} + \ldots + 2 c_{n} e^{int}
    \end{equation*}

    
    Now, we claim that every function $f \in C\left( \T , \R \right)$ is a uniform limit of a trignometric polynomial of the form \ref{eqn:trig-poly-1}. We will be done if we show that the negative Fourier coefficients of real valued function is the conjugate of the its positive counterpart, that is, for each $n\in \Z_{\ge 0}$, we have that $\hat{f}\left( -n \right) = \overline{\hat{f} \left( n \right)}$. To show this, take any $n\in \Z _{\ge 0}$ and then observe that
    \begin{align*}
	\hat{f} \left( -n \right) &= \int_{\T} f\left( e^{it} \right) e^{int} \frac{dt}{2\pi} \\
	&= \overline{\int_{\T} f\left( e^{it} \right) e^{-int} \frac{dt}{2\pi}} \\
&= \overline{\hat{f} \left( n \right)}
    \end{align*}

This shows that the Cesaro means of a real valued function is a trignometric polynomial of the form \ref{eqn:trig-poly-1} and since the Cesaro means converges to $f$ uniformly. Thus, the closure of the real parts of $\calA$ is indeed $\calC \left( \T , \R \right)$.

Now, let $\mu$ be a finite signed measure on $\T$ such that $\int f \, d\mu =0$ for every $f \in \calA$. We show that $\mu$ is the zero measure. Notice that if $f\in \calA$ then
\begin{equation*}
    0=\int_{\T} f \, d\mu = \int_{\T} \Re (f) \, d\mu + i\int_{\T} \Im (f) \, d\mu
\end{equation*}
Hence, it follows that 
\begin{equation*}
    \int_{\T} \Re (f) \, d\mu = 0
\end{equation*}
for every $f\in \calA$.
Now, if $g\in \calC \left( \T, \R \right)$ then by the first part of this theorem, there is a sequence $\left\{ f_{n} \right\} \in \calA$ such that $\Re \left( f_{n} \right) $ converges to $g$ uniformly. By the Dominated Convergence Theorem (which holds, thanks to Jordan Decomposition Theorem), we have that $\int_{\T} g \, d\mu = 0$. 

Now to prove that every $\mu = 0$, it suffices to show that $\hat{\mu} \left( n \right) = 0$ for every $n\in \Z$\footnote{See Page 41, Corollary 2.3 of Mashreghi.}. Now, notice that for any $n\in \Z$, we have
\begin{align*}
    \hat{\mu} (n) &= \int_{\T} e^{-int} d\mu \left( e^{it} \right) \\
    &=  \int_{\T} \left(\cos \left( nt \right) - i \sin \left( nt \right) \right) d\mu \left( e^{it} \right) \\
    &= \int_{\T} \cos \left( nt \right) d\mu \left( e^{it} \right) - i\int_{\T} \sin \left( nt \right) d\mu \left( e^{it} \right) \\
    &= 0
\end{align*}
This completes the proof!
\end{proof}

\begin{corollary}
    Let $\mu$ be a finite signed  measure on $\T$ such that $\int f d\mu = 0$ for every $f\in \calA$ which vanishes at the origin then $\mu$ is a constant multiple of Lebesgue measure.
    \label{cor:constant-multiple-of-Lebesgue}
\end{corollary}
\begin{proof}
    We first prove the following claim: If $f \in \calA$ then $\int_{\T} f d\mu = \frac{f\left( 0 \right)}{2\pi}$. Since the negative Fourier cofficients are zero and $f$ is continuous, we have that the Cesaro means converge uniformly to $f$, that is,
    \begin{align*}
	\sum_{k=0}^{\infty} \hat{f} \left( n \right) e^{int} \rightarrow f \text{ uniformly}
    \end{align*}
    Thus,
\begin{align*}
    \int_{T} f\left( e^{it} \right) \frac{dt}{2\pi} &= \frac{1}{2\pi} \int_{\T}  \left(  \sum_{k=0}^{\infty}\hat{f} \left( n \right) e^{int} \right) dt\\
    &= \frac{1}{2\pi} \sum_{k=0}^{\infty} \int_{\T} \left( \hat{f} \left( n \right) e^{int} dt \right) \\
    &= \frac{\hat{f}(0)}{2\pi} 
    \end{align*}

    Now, we proceed to the proof. We define a measure $d\nu = d\mu - \frac{1}{2\pi} \mu \left( \T \right) dt$. Now, we have that
    \begin{align*}
	\int_{\T} f\left( e^{it} \right) d\nu \left( e^{it} \right) &= \int_{\T}[f-f\left( 0 \right)] \left( e^{it} \right) d\mu \left( e^{it} \right) + f\left( 0 \right) \int_{\T} d\nu \left( e^{it} \right) \\
	&= 0
    \end{align*}
    Hence, we have that $d\mu = \frac{1}{2\pi} \mu \left( \T \right) dt$.
\end{proof}

We will be working entirely on the $\T$. So, $\calA$ and $H^{2}$ will be the spaces on the unit circle rather on the open unit disc. 

Now, consider $\calA$ as a subset of $L^{2} \left( \T , \scrB \left( \T \right) , \mu \right)$ where $\mu$ is any finite positive measure. Let $\calA _{0} = \left\{ f \in \calA : \int_{\T} f\left( e^{it} \right) \frac{dt}{2\pi} = \frac{\hat{f} (0)}{2\pi}=0 \right\}$. It is easily seen that $\calA _{0}$ is a subspace of $L^{2} \left( d\mu \right)$. Therefore, we have the closed subspace spanned by $\calA$ is $\llbracket \calA_{0} \rrbracket = \overline{ \operatorname{span} (\calA_{0}) } = \overline{\calA_{0}}$. By a theorem of Hilbert spaces, we have that there is some vector $F \in \llbracket \calA _{0} \rrbracket$ such that
\begin{equation*}
    \inf_{f\in \llbracket \calA _{0} \rrbracket} \int \abs{1-f^{2}} d\mu = \int \abs{1-F^{2}} d\mu
\end{equation*}

But since $d\left( 1, \llbracket \calA_{0}\rrbracket \right) = d \left( 1, \overline \calA_{0} \right) \stackrel{\footnote{this is true for any set in a metric space}}{=} d \left( 1, \calA_{0} \right)$, we have

\begin{equation*}
    \inf_{f\in \calA _{0}} \int \abs{1-f^{2}} d\mu = \int \abs{1-F^{2}} d\mu
\end{equation*}

Note that this $F$ is the orthogonal projection of $1$ into the closed subspace spanned by $\calA_{0}$.

\begin{theorem}
    Let $\mu$ be a finite positive Borel measure on $\T$ and suppose that the constant function $1$ is not in $\llbracket \calA_{0} \rrbracket$. Then let $f=P_{\llbracket \calA_{0} \rrbracket} \left( 1 \right)$. Then the following holds:
    \begin{enumerate}
	\item The measure $d\nu = \abs{1-F^{2}} d\mu$ is a nonzero constant multiple of the Lebesgue measure. In particular, Lebesgue measure is absolutely continuous with respect to $\mu$.
	\item The function $\left( 1-F \right)^{-1} \in H^{2}$.
	\item If $h= \left( \frac{d\mu}{d\theta} \right)$ then $\left( 1-F \right)h \in L^{2} = L^{2} \left( \frac{d\theta}{2\pi} \right)$.
    \end{enumerate}
    \label{thm:}
\end{theorem}

\begin{proof}
    \begin{enumerate}
	\item Let $S=\llbracket \calA _{0} \rrbracket$. We begin to prove part one of the theorem. Let $F= P_{S} \left( 1 \right)$. Then we have by the uniqueness of the decomposition that
    \begin{equation*}
	1=\underbrace{F}_{P_S (1)} + \underbrace{1-F}_{P_{S^{\perp} (1)}}
    \end{equation*}

    Thus, we have that $\left( 1-F \right)$ is orthogonal to every element in $S$ and hence, in particular, any element in $\calA _{0}$ (because $\calA_{0} \subset S \leadsto S^{\perp} \subset \calA_{0} ^{\perp}$). We claim that $1-F$ is orthogonal to $\left( 1-F \right) f $ for every $f\in \calA _{0}$.  But before, we do this, we need to show that $(1-F)f \in L^{2} (d\mu)$. Observe that\begin{align*}
	\int_{\T} \abs{(1-F)f^{2}} d\mu \le \norm{f}_{\infty}^{2} \norm{1-F}_{2}^{2} < \infty
    \end{align*}

    To prove this, note that we showed that $S=\overline{\calA_{0}}$ in the paragraph before the statement of this theoremand since $F \in S$, there is a sequence $\left\{ f_{n} \right\} \in \calA _{0}$ converging to $F$. Hence, we have that $\left\{ f\left( 1-f_{n} \right) \right\}$ is a sequence in $\calA _{0}$\footnote{$\calA_{0}$ is an algebra!} converges to $f\left( 1-F \right)$ in the $L^{2}$-norm. Hence, we have that

    \begin{align*}
	\ip{f(1-F), (1-F)} &= \ip{\lim_{n\to \infty} f\left( 1-f_{n} \right), \left( 1-F \right)} \\
	&= \lim_{n\to \infty} \ip{f\left( 1-f_{n} \right), \left( 1-F \right)} &\text{continuity of the inner product} \\
	&= 0  & \text{$1-F$ is orthogonal to $\calA_{0}$}
    \end{align*}
     
    Now, let $d\nu = \abs{1-F}^{2} d\mu$. We have shown that for any $f\in \calA_{0}$, 
    \begin{align*}
	\int_{\T} f d\nu &= \int_{\T} f \abs{1-F}^{2} d\mu = \ip{f(1-F), (1-F)} = 0
    \end{align*}
    Hence, by Corollary \ref{cor:constant-multiple-of-Lebesgue}, we have that $d\mu = k\, d\lambda$ for some $k \ge 0$.

    Now, we claim that this $k\ne 0$. If $k=0$ then we would have that 
    \begin{equation*}
	\int_{\T} d\nu = 0 \leadsto \int_{\T} \abs{1-F}^{2} d\mu = 0
    \end{equation*}
    Hence, we have that $F=1$ $\mu$-almost everywhere\footnote{See Corollary 2.3.12 of Cohn's Measure Theory.}. But then we have that $1\in S$ which contradicts our assumption. Hence $k \ne 0$.
\item Observe that part 1 of the theorem tells us that 
    \begin{equation*}
	\abs{1-F}^{2} d\mu = k\, d\lambda \text{ where } k \ne 0
    \end{equation*}
    Then we have that by Lebesgue Decomposition Theorem
    \begin{align*}
	d\mu = h d\lambda + d\mu_{s}
    \end{align*}
    for some positive $\calL^{1}$-function $h$ and some singular measure $\mu_{s}$.
    Hence, we have that
    \begin{align}
	\abs{1-F}^{2} d\mu &= \abs{1-F}^{2} \left( h d\lambda + d\mu_{s} \right) \\
	&= \underbrace{\abs{1-F}^{2} h d\lambda}_{(1)} + \underbrace{\abs{1-F}^{2} d\mu_{s}}_{(2)}
	\label{eqn:decomposed-measure}
    \end{align}
    By the uniqueness of the Lebesgue Decomposition Theorem (one needs to verify that the measures obtained in (1) and (2) are absolutely continuous and singular), we have that
\begin{align*}
    \abs{1-F}^{2} h &= k \text{ $\lambda$-almost everywhere.} \\
    \leadsto \frac{1}{\abs{1-F}^{2}} &= \frac{h}{k} \text{ $\lambda$-almost everywhere.}
\end{align*}
This tells us that $(1-F)^{-1} \in L^{2} \left( d\lambda \right)$ where $\lambda$ is the Lebesgue measure on $\T$. Also, note that $F=1$ $d\mu_{s}$-almost everwhere. Hence, we have that 
\begin{equation}
    \abs{1-F}^{2} d\mu = \abs{1-F}^{2} h d\lambda = k d\lambda \leadsto d\mu = \abs{1-F}^{-2} k d\lambda
    \label{eqn:prelude-to-f-m}
    \end{equation}

    Now, we proceed to show that the negative Fourier coefficients of $\left( 1-F \right)^{-1}$ is 0 then we would have shown that $\left( 1-F \right)^{-1} \in H^{2}$. Consider the following for $f\in \calA_{0}$:
\begin{align*}
    k \int_{\T} \left( 1-F \right)^{-1} f d\lambda &= k \int_{\T} \left( 1-\overline F \right) f \abs{1-F}^{-2} d\lambda \\
    &= k \int_{\T} \left( 1-\overline F \right) f d\mu \\
    &= 0
\end{align*}
The last line follows from the fact that $1-F \perp f$ for any $f\in \calA_{0}$. Hence this holds for any $e^{in\theta}$ and hence we are done.

\item Note that the derivative of $\mu$ with respect to normalized Lebesgue meaures is $h$ as demonstrated in Equation \ref{eqn:decomposed-measure}. Now, Equation \ref{eqn:prelude-to-f-m} shows that 
    \begin{align*}
	\abs{1-F}h = k\abs{1-F}^{-1} \text{ $\lambda$-almost everywhere.}
    \end{align*}
    Since $\abs{ 1-F}^{-1} \in L^{2} \left( \lambda \right)$, so is $(1-F)h$.
    \end{enumerate}
\end{proof}

\section{Szegö's Theorem}
